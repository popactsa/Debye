\section{Введение}
\qquadСоздание термоядерного реактора требует исследования тепловых нагрузок на
внутреннюю поверхность установки и моделирования движения филаментов. ПЛМ,
возникающие во всех конфигурациях установок магнитного удержания плазмы имеют
аналогичные свойства, одним из которых является резкое(время развития ПЛМ
составляет порядка $50-100$ мкс) сильное увеличение теплового
потока(\hl{сколько?}). В перспективных и некоторых существующих токамаках
наблюдается плавление материала, что в случае проплавления материала моноблока
является неприемлимым для работы токамака~\cite{gunn2017surface}. Анализ
подобных выбросов и их влияния на параметры дебаевского слоя является одной из
главных задач на пути к созданию стабильных в работе и коммерчески успешных
термоядерных установок с магнитным удержанием плазмы.

Решение задачи моделирования динамики плазмы требует определения параметров
дебаевского слоя --- граничного условия. Существующие модели, как правило, не
учитывают наличие осцилляций магнитного поля, возникающих в дебаевском слое при
прохождении филаментов по поверхности. Анализ характера изменения параметров
дебаевского слоя и осцилляций магнитного поля в филаменте позволяет предполагать
увеличение значения усреднённого теплового потока на поверхность стенки. Так же,
ввиду изменения разности потенциалов в слое(плавающего потенциала) возникает
ненулевой ток, и как следствие, ненулевое сопротивление дебаевского слоя.
\pagebreak

\section{Обозначения и сокращения}
\begin{align*}
	\theta(x) &\text{ --- функция Хевисайда}\\
	\text{ПЛМ} &\text{ --- пограничная локализованная мода}\\
	a &\text{ --- атомы}\\
	c_s &\text{ --- скорость звука в плазме}\\
	e &\text{ --- электроны}\\
	i &\text{ --- ионы}\\
	se &\text{ --- вход в дебаевский слой}\\
	te &\text{ --- термоэлектронная эмиссия}\\
	th &\text { --- тепловая скорость}\\
	trans &\text { --- параметр при смене режимов работы дебаевского слоя}\\
	s &\text{ --- поверхность плитки}\\
	V_f &\text{ --- плавающий потенциал}\\
	W &\text{ --- вольфрам}
\end{align*}
\pagebreak

\section{Цели и задачи}
\qquadЦель работы --- проведение анализа динамических характеристик нестационарного дебаевского слоя. Задачами являются:
\begin{itemize}
	\item Составление модели дебаевского слоя с учётом наличия термоэлектронной эмиссии с поверхности стенки
	\item Проведение численного моделирования параметров дебаевского слоя с использованием полученной модели и учётом теплопроводности в материале стенки
	\item Составление вывода о характере эволюции проводимости дебаевского слоя в присутствии осцилляций потенциала на входе в слой
	\item Проведение численного моделирования прохождения филамента по поверхности стенки с применением полученной модели в качестве граничного условия, сравнение с результатами 
		моделирования с использованием других моделей
	\item Составление вывода о значении тепловой нагрузки на стенку, сравнение с классическими значениями
\end{itemize}
\pagebreak
\pagebreak

