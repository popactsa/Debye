\subsection{Режим ограничения объёмным зарядом}

\qquadС увеличением электронной эмиссии со стенки, начиная с некоторого критического значения, 
возникает виртуальный катод --- минимум потенциала, где как следствие, отсутствует поле.
Возникающая структура регулирует ток частиц в обе стороны, в частности, возникает отражение
эмитированных электронов на разности потенциалов $\varphi_{vc} - \varphi_s < 0$ 
обратно на стенку. Таким образом, в системе уравнений возникает дополнительная 
неизвестная --- $\varphi_{vc}$.

\qquadТок прошедших виртуальный катод электронов из плазмы определяется соответствующим током 
на виртуальном катоде, т.к. между стенкой и виртуальным катодом электроны отражаются на 
стенку.

\qquadДля потока электронов в области $x > x^{vc}$ верно уравнение неразрывности:
\begin{equation}
	dj_{te}(\upsilon_s) = e\upsilon_s n_{te}^s \cdot f(\upsilon_s) d\upsilon_s = e\upsilon(x, \upsilon_s)\cdot dn_{te}(x)
\end{equation}

\qquadОтсюда:
\begin{equation}
	n_{te} = n_{te}^s \int\limits_{\upsilon_{vc}}^{\infty}\,d\upsilon_s \cfrac{\upsilon_s f(\upsilon_s)}{\upsilon}
	\label{n_thermionic}
\end{equation}
где $\upsilon_{vc}$ --- минимальная необходимая для преодоления виртуального катода 
при движении со стенки скорость электронов.

% \qquadСтоит отметить, что $n_{te}^s$ в случае экранирования объёмным зарядом(как и в классическом 
% случае) является концентрацией электронов в потоке, эмитируемом со стенки.

\qquadТ.к. слой считается бесстолкновительным, из закона сохранения энергии выразим скорость  
электронов в точке --- $\upsilon$:
\begin{equation*}
	\cfrac{m_e(\upsilon_s)^2}{2} - e(V_f + \varphi_{se}) = \cfrac{m_e\upsilon^2}{2} - e\varphi
\end{equation*}

\begin{equation}
	\upsilon = \upsilon_s\sqrt{1 + \cfrac{2e(\varphi - (V_f + \varphi_{se}))}{m_e(\upsilon_s)^2}}
	\label{upsilon_th}
\end{equation}

\qquadИнтегрируя, получим:
\begin{equation}
	n_{te}(\varphi) = n_{te}^s\erfc{\sqrt{\cfrac{e(\varphi - \varphi_{vc})}{T_s}}}\exp{\cfrac{e(\varphi - (V_f + \varphi_{se}))}{T_s}}
\end{equation}

В области между поверхностью плитки и виртуальным катодом, примем, что ансамбль термоэлектронов находится в равновесии и 
распределение является максвелловским.

Составим систему уравнений аналогичную полученной в классическом режиме: неизвестными приняты $n_e^{se}, \varphi_{se}, \left(\cfrac{\partial \varphi}{\partial x}\right)_s, V_f, V_{vc}$,
заданными константами --- $n_i^{se}, T_s$.

Область между виртуальным катодом и входом в дебаевский слой можно рассматривать 
как при классическом режиме: основное отличие заключается в том, что выражение для 
термоэмиссионного тока теперь содержит множитель ''отсечки''. Таким образом, заменив 
в выражениях $V_f$ на $V_{vc}$ и учтя отражение термоэмиссионных электронов, получим 
аналог прежних четырёх уравнений системы.

Замыкающее уравнение может быть получено интегрированием уравнения Пуассона в области 
между поверхностью плитки и виртуальным катодом:
\begin{equation}
    \begin{split}
        \left(\cfrac{d\varphi}{dx}\right)^2_s &= 8\pi\left[n_im_i\upsilon_0^2\sqrt{1 - \cfrac{2e(\varphi - \varphi_{se})}{m_i\upsilon_0^2}} + \right.
        \\ &+ n_e^{se}T_e\exp{\cfrac{-e\varphi_{se}}{T_e}}\left(\exp{\cfrac{e\varphi}{T_e}}\erfc{\sqrt{\cfrac{e(\varphi - (V_{vc} + \varphi_{se}))}{T_e}}} \right.+ 
        \\ &+ \left. \cfrac{2}{\sqrt{\pi}}\exp{\cfrac{e(V_{vc} + \varphi_{se})}{T_e}}\sqrt{\cfrac{e(\varphi - (V_{vc} + \varphi_{se}))}{T_e}}\right) + 
        \\ &+ \left. n_{te}^sT_s\exp{\cfrac{e(\varphi - (V_f+\varphi_{se}))}{T_s}}\right]\Bigg|_{V_{vc} + \varphi_{se}}^{V_f + \varphi_{se}}
    \end{split}
    \label{eq::Poisson_SCL_alpha}
\end{equation}

Выражение для теплового потока имеет вид:
\begin{equation}
    q = n_i^{se}\upsilon_0\left(\cfrac{m_i\upsilon_0^2}{2} - eV_{vc}\right) + \cfrac{1}{4}n_e^{se}\upsilon_e^{th}\cdot2T_e\exp\left[\cfrac{eV_{vc}}{T_e}\right]
\end{equation}


